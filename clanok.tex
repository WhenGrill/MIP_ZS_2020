\documentclass[10pt,twoside,slovak,a4paper]{article}

\usepackage[slovak]{babel}
%\usepackage[T1]{fontenc}
\usepackage[IL2]{fontenc}
\usepackage[utf8]{inputenc}
\usepackage{graphicx}
\usepackage{url} % príkaz \url na formátovanie URL
\usepackage{hyperref} % odkazy v texte budú aktívne (pri niektorých triedach dokumentov spôsobuje posun textu)

\usepackage{cite}
%\usepackage{times}

\pagestyle{headings}

\title{Aplikácie a riešenia dištančného vzdelávania a e-vzdelávania\thanks{Semestrálny projekt v predmete Metódy inžinierskej práce, ak. rok 2020/21, vedenie: Ing. Fedor Lehocki, PhD.}}

\author{Lukáš Štrbo\\[2pt]
	{\small Slovenská technická univerzita v Bratislave}\\
	{\small Fakulta informatiky a informačných technológií}\\
	{\small \texttt{xstrbol@stuba.sk}}
	}

\date{\small 15. október 2020}



\begin{document}

\maketitle

\begin{abstract}
E-vzdelávanie sa stáva stále viac populárnejšou metódou nadobúdania vedomostí. Mnohí ľudia ju preferujú najmä kvôli rýchlosti a efektívnosti získavania poznatkov.
Prostredníctvom internetu sa dokážeme vzdelávať pomocou rôznych aplikácií, webov, kurzov alebo aj diskusných fór.
S e-vzdelávaním ide ruka v ruke dištančné vzdelávanie, ktoré hlavne v ťažších časoch, ako je napríklad nemožnosť zúčastňovať sa prezenčnej výučby z dôvodu pandémie COVID-19, 
 je voľbou číslo jedna. Cieľom tejto práce je sprehľadniť čitateľovi rôzne metódy dištančného vzdelávania. Rozoberieme si a porovnáme riešenia dištančného vzdelávania a ich priamu
 aplikáciu. Zameriame sa na výhody a nevýhody, ale aj ktoré softvéry alebo platformy sú lepšie pre dištančné vzdelávanie v praxi. 
 V rámci porovnávania sa zameriame aj na efektivitu a aplikáciu daných metód dištančného vzdelávania.
\end{abstract}



\section{Úvod}

%\acknowledgement{Ak niekomu chcete poďakovať\ldots}



\bibliography{zdroje}
\bibliographystyle{plain}
\end{document}
